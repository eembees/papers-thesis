\section{Clockwork Variational Autoencoders}
The clockwork VAE \cite{saxena_clockwork_2021} aims to learn higher-level, abstract prediction timelines \textit{without needing to predict the actual images going forward.}
They term this "Temporally Abstract Latent Dynamics Models".


\subsection{CW-VAE Architecture and components}
The CW-VAE is composed of a hierarchy of "states" as seen in \cref{fig:cwvae-architecture}

\begin{figure}[hb]
    \begin{small}
        \begin{center}
            \includegraphics[width=0.95\textwidth]{figures/cwvae-architecture.png}
        \end{center}
        \caption{
            \textbf{Arrows:} Solid is the generative model, dashed is the inference model. 
            \textbf{Left:} The general setup of the CW-VAE, with a temporal abstraction factor \(k=2\). 
            The upper state only updates every kth timestep, and this would compound in higher hierarchies. 
            As such we see that the video frame with \(t=2\) still will feed into \(s^2_1\), and likewise the frame at \(t=4\) feeds into \(s^2_3\)
            \textbf{Right:} The internals of states \(s_l^t\). 
            }
        \label{fig:cwvae-architecture}
    \end{small}
\end{figure}

\subsubsection{The latent states \(s^l_t\)}

The latent state is composed of a deterministic variable \(h_t\) and a stochastic variable \(z_t\). 


\paragraph{Updates to latent states during inference}
The latent states are updated at the "active" timesteps 
\(\mathcal{T}_l = \{ t \in [1, T] | t \mathrm{mod} k^{l-1} = 1 \}\), 
where \(k\) is the dilation factor.
This means that latent states will only be updated at each \(k_{l-1}\) timestep, where \(l\) is the "level" of the latent variable.
Otherwise, the states are copied from the previous timestep.

During inference, all "active" latents will receive a CNN image embedding (in \cref{fig:cwvae-architecture} this is the "bottom-up observations").
The posterior \(q^l_t\) for that latent is calculated 
"as a function (what function?) of the input features, 
\hl{(ASK JAKOB: Is this the arrow from $z$ to the combination node?) } 
the posterior sample at the previous step (temporal context), 
and the posterior sample above (top-down context)"
The posterior / "Gaussian Belief" \(q^l_t\), which is a diagonal Gaussians with means and variances predicted from the deterministic variable.


The deteministic variable is updated with a GRU at every active step.

\subsubsection{Embeddings and layers in between.}

The authors (Appendix C) claim to have used "architectures very similar to the DCGAN".

The DCGAN paper's decoder architecture is shown in \cref{fig:cwvae-dcgan-arch}. \cite{radford_unsupervised_2016}

\begin{figure}[hb]
    \begin{small}
        \begin{center}
            \includegraphics[width=0.95\textwidth]{figures/cwvae-dcgan-arch.png}
        \end{center}
        \caption{\textit{From DCGAN paper: }
        A 100 dimensional uniform distribution Z is projected to a small spatial extent convolutional representation with many feature maps.
        A series of four fractionally-strided convolutions (in some recent papers, these are wrongly called
        deconvolutions) then convert this high level representation into a 64 × 64 pixel image. Notably, no
        fully connected or pooling layers are used.}
        \label{fig:cwvae-dcgan-arch}
    \end{small}
\end{figure}


